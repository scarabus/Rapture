The Bootstrap API manages the configuration of the repositories that are used to store
configuration and settings information in \Rapture.

\begin{figure}[H]
\centering
\begin{tikzpicture}
\node[external](Bootstrap) { Bootstrap     };
\node[api] (Settings) [below=of Bootstrap] { Settings}
   edge [post] (Bootstrap);
\node[api] (Config) [left=of Settings] { Configuration }
   edge [post, bend left=45] (Bootstrap);
\node[api] (Ephem) [right=of Settings] { Ephemeral }
   edge [post,bend right=45]  (Bootstrap);
\node[ctx] (Sessions) [ below=of Ephem] { Sessions }
      edge [line] (Ephem);
\node[ctx] (Users) [ below=of Settings] { Users }
   edge [line] (Settings);
\node[ctx] (Repos) [ below=of Config] { Repos }
   edge [line] (Config);
\node[ctx] (Workflows) [ below=of Repos] { Workflows }
      edge [line] (Repos);
\node[ctx] (D1) [ below=of Workflows] { ... }
      edge [line] (Workflows);
\node[ctx] (D2) [ below=of Users] { ... }
      edge [line] (Users);
\node[ctx] (D1) [ below=of Sessions] { ... }
      edge [line] (Sessions);
\end{tikzpicture}
\caption { Bootstrap and configuration repositories }
\end{figure}

When a \Rapture environment starts it initially looks for the "bootstrap" repository. The configuration for
this repository (as defined in the \verb+createDocRepo+ call in the document API) is stated in the
configuration file/resource for \Rapture called \verb+rapture.cfg+. This file is usually on the classpath of
a \Rapture environment and it needs to be identical on all \Rapture servers.

The bootstrap repo contains three documents - these documents define the configuration of three other
document repositories. The configuration repo contains all of the definitions of entities within \Rapture --
repositories, workflows, schedules, audit logs and the like. When you create a document repository for your use \Rapture
stores the definition of that configuration in this repository.

The settings repository contains information about users. When you create a user in \Rapture the definition of that
user is stored in the settings repository.

Finally the ephemeral repository is used to store cached or temporary data. Typically this contains the session information needed
when a user logs into \Rapture.

This two stage approach to defining the locations of configurations gives a developer (the designer of a \Rapture environment) the
ability to differentiate between the underlying technology used to store configuration data. For example the ephemeral repository could
be hosted on a \verb+REDIS+ based repository, the settings on \verb+Postgres+ (and be versioned) and the configuration on \verb+MongoDB+.

The contents of these repositories can be navigated and accessed at the lower level through the Sys API.
