The \verb+addTransition+ call is used to add a transition to an existing workflow definition.

A transition is a link from one step to another step that will be traversed by a workflow if
the return value upon execution of a step matches the transition name.

As an example, imagine a step (implementing using a \Reflex script) that could return two results --
"filesProcessed" and "error". For this step we would add two transitions. The first would map the result
"filesProcessed" to a step that would perhaps continue the execution of the workflow. The second would map the
result "error" to a terminal step or maybe an error handler step.

The transition parameter to this call is a complex type which has fields defined as below.

\begin{table}[h]
  \small
\begin{center}
\begin{tabular}{r l p{8cm}}
  Field & Type & Description \\
  \hline
  name & string & The name of this transition in this step.\\
  targetStep & string & The step that this transition points to.\\
\end{tabular}
\end{center}
\end{table}

The \verb+targetStep+ field of a transition can contain special directives that control
special behavior in the workflow engine. These directives are described below.

\begin{table}[h]
  \small
\begin{center}
\begin{tabular}{r p{8cm}}
  Directive & Meaning \\
  \hline
  \verb+$RETURN+ & This directive instructs the workflow to exit and return a value to
     the caller. The return value string is supplied after the \verb+$RETURN+ field like so:
     \verb+$RETURN:ok+. \\
  \verb+$JOIN+ & What what \\
  \verb+$FORK+ & What what \\
  \verb+$SPLIT+ & What what \\

\end{tabular}
\end{center}
\end{table}
